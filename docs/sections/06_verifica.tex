\section{Verifica}
Ci sono varie verifiche che si possono fare andando a controllare se tutta una serie di caratteristiche volute si riflettano veramente nel modello simulativo costruito e quindi nei risultati ottenuti. In particolare, andrò ad effettuare le seguenti verifiche:
\begin{itemize}
    \item Verifica del corretto funzionamento di routing basato su soglia si \(SI_{max}\).
    \begin{itemize}
        \item Controllo che impostando \(SI_{max}=0\) venga utilizzato solo lo spike server.
        \item Controllo che impostando \(SI_{max}=\infty\) non venga mai utilizzato lo spike server.
    \end{itemize}
    \item Verifica delle metriche prestazionali:
    \begin{itemize}
        \item Verifico che aumentando \(SI_{max}\) l'utilizzazione del web server aumenta
        \item Verifico che il throughput sia pari al tasso di arrivo quando il sistema è stabile e uguale al tasso di servizio quando il sistema è sovraccarico.
    \end{itemize}
    \item Verifica della distribuzione
    \begin{itemize}
        \item Verifica che media e cv della distribuzione iperesponenziale corrispondano ai valori impostati.
    \end{itemize}
\end{itemize}
    
\subsection{Verifica del routing basato su soglia \(SI_{max}\)}
\subsubsection{Verifica con \(SI_{max}=0\)}
Impostando \(SI_{max}=0\) si può verificare che tutti i job vengano instradati allo spike server. Infatti, come mostrato in figura \ref{fig:verifica_si_0}, l'utilizzazione del web server è nulla, mentre tutto il carico viene gestito dallo spike server.

\begin{figure}[H]
    \centering
    \includegraphics[width=0.8\textwidth]{images/verifica_si_0.png}
    \caption{Verifica routing con \(SI_{max}=0\)}
    \label{fig:verifica_si_0}
\end{figure}
\subsubsection{Verifica con \(SI_{max}=\infty\)}
Impostando \(SI_{max}=\infty\) si può verificare che nessun job venga instradato allo spike server. Infatti, come mostrato in figura \ref{fig:verifica_si_inf}, l'utilizzazione dello spike server è nulla, mentre tutto il carico viene gestito dal web server.

\begin{figure}[H]
    \centering
    \includegraphics[width=0.8\textwidth]{images/verifica_si_inf.png}
    \caption{Verifica routing con \(SI_{max}=\infty\)}
    \label{fig:verifica_si_inf}
\end{figure}

\subsection{Verifica delle metriche prestazionali}
\subsubsection{Verifica dell'utilizzazione al variare di \(SI_{max}\)}
Impostando un carico di lavoro fisso e variando il valore di \(SI_{max}\), si può verificare che l'utilizzazione del web server aumenti al crescere di \(SI_{max}\), come mostrato in figura \ref{fig:verifica_utilizzazione_si}.

\begin{figure}[H]
    \centering
    \includegraphics[width=1\textwidth]{images/verifica_utilizzazione_si.png}
    \caption{Verifica dell'utilizzazione del web server al variare di \(SI_{max}\)}
    \label{fig:verifica_utilizzazione_si}
\end{figure}

\begin{table}[H]
  \begin{center}
    \caption{Verifica dell'utilizzazione del web server al variare di \(SI_{max}\)}
    \pgfplotstabletypeset[
      multicolumn names, % allows to have multicolumn names
      col sep=comma, % the seperator in our .csv file
      display columns/0/.style={
        column name={$SI_{max}$},
        column type={S[round-mode=places, round-precision=0]},string type},  % use siunitx for formatting
      display columns/1/.style={
        column name={$\text{U}_{web}$},
        column type={S[round-mode=places, round-precision=4]},string type},
      display columns/2/.style={
        column name={$\text{U}_{web}$ CI95},
        column type={S[round-mode=places, round-precision=4]},string type},
      every head row/.style={
        before row={\toprule}, % have a rule at top
        after row={\midrule}, % rule under units
            },
        every last row/.style={after row=\bottomrule}, % rule at bottom
    ]{data_tables/graph_data_si_max.csv} % filename/path to file
  \end{center}
\end{table}

\subsubsection{Verifica del throughput}
Variando il carico di lavoro da 1 req/s a 12 req/s e impostando \(SI_{max} = \infty\), in modo da misurare solo il web server, si può verificare che il throughput del sistema non superi mai il suo limite teorico, che nel caso considerato è uguale a \(\min(\mu_{web}, \lambda_{web})\). Infatti, come evidenziato in Figura \ref{fig:wrong_throughput}, il throughput misurato \(X_{measured} = N_{completed} / T_{sim}\) non supera il limite teorico e si mantiene sempre al di sotto di esso.

\begin{figure}[H]
    \centering
    \includegraphics[width=1\textwidth]{images/wrong_throughput.png}
    \caption{Throughput misurato che supera il valore teorico}
    \label{fig:wrong_throughput}
\end{figure}

\begin{table}[H]
  \begin{center}
    \caption{Verifica del Throughput}
    \label{tab:verifica_throughput2}
    \pgfplotstabletypeset[
      multicolumn names, % allows to have multicolumn names
      col sep=comma, % the seperator in our .csv file
      display columns/0/.style={
        column name={$\lambda$},
        column type={S[round-mode=places, round-precision=0]},string type},  % use siunitx for formatting
      display columns/1/.style={
        column name={$X_{measured}$},
        column type={S[round-mode=places, round-precision=4]},string type},
      display columns/2/.style={
        column name={$X_{measured}$ CI95},
        column type={S[round-mode=places, round-precision=4]},string type},
      display columns/3/.style={
        column name={$X_{theoretical}$},
        column type={S[round-mode=places, round-precision=4]},string type
        },
      every head row/.style={
        column name={},
        before row={\toprule}, % have a rule at top
        after row={\midrule}, % rule under units
            },
        every last row/.style={after row=\bottomrule}, % rule at bottom
    ]{data_tables/data_throughput_si_max_inf.csv} % filename/path to file
  \end{center}
\end{table}

\subsection{Verifica della distribuzione iperesponenziale}
Per verificare che la distribuzione iperesponenziale implementata nel simulatore corrisponda a quella voluta, sono state effettuate delle misurazioni della media e del coefficiente di variazione dei tempi di servizio generati, confrontandoli con i valori teorici impostati. In particolare sono state effettuate sempre 100 repliche di simulazione con \(SI_{max} = 80\). Effettivamente, come mostrato in figura \ref{fig:verifica_distribuzione}, i valori teorici rientrano perfettamente negli intervalli di confidenza al 95\% delle misurazioni effettuate.

\begin{figure}[H]
    \centering
    \includegraphics[width=1\textwidth]{images/verifica_distribuzione.png}
    \caption{Verifica della distribuzione iperesponenziale dei tempi di servizio}
    \label{fig:verifica_distribuzione}
\end{figure}