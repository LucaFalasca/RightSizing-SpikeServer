\section{Validazione}
Dato che in questo studio non sono disponibili dati reali con cui confrontare i risultati ottenuti dalla simulazione, la validazione del modello simulativo confronterà i risultati ottenuti con quelli ottenuti dallo studio di riferimento nella prossima sezione del report.


\subsection{Confronto tempi di risposta}
Nello studio di riferimento vengono effettuati dei test sui tempi di risposta sul webserver e sullo spike server. 

\begin{figure}[H]
    \centering
    \includegraphics[width=0.7\textwidth]{images/response_time_serazzi.png}
    \caption{Tempi di risposta del webserver e dello spike server in funzione del carico. Fonte: \cite{serazzi_spike_server}.}
    \label{fig:response_time_serazzi}
\end{figure}

\begin{figure}[H]
    \centering
    \includegraphics[width=0.7\textwidth]{images/response_time_web_vs_si_max.png}
    \caption{Tempi di risposta del webserver in funzione del SI\_max ottenuti dalla simulazione.}
    \label{fig:response_time_web_sim}
\end{figure}

\begin{figure}[H]
    \centering
    \includegraphics[width=0.7\textwidth]{images/response_time_spike_vs_si_max.png}
    \caption{Tempi di risposta dello spike server in funzione del SI\_max ottenuti dalla simulazione.}
    \label{fig:response_time_spike_sim}
\end{figure}

Controntandoli con i risultati ottenuti dalla mia simulazione, si nota come l'andamento sia essenzialmente lo stesso, tuttavia i valori dei tempi differiscono. Questo è probabilmente dovuto dal tempo di simulazione che Serazzi non esplicita direttamente e quindi probabilmente differente dai 5000 secondi utilizzati nella mia simulazione. Tuttavia per non rendere troppo lunghe le simulazioni, ho eseguito una simulazione con SI_max = 160 su 50 repliche anziché 100 e aumentato il tempo di simulazione a 20000 secondi, per dimostrare che il motivo della differenza nei tempi di risposta è dovuto proprio al tempo di simulazione.