\documentclass{article}
\usepackage[utf8]{inputenc}
\usepackage{graphicx}
\usepackage{float}
\usepackage{microtype}
\usepackage{listings}
\usepackage[a4paper,
            top=3cm,
            bottom=3cm,
            inner=3cm,      % Margine interno (lato rilegatura)
            outer=2cm,      % Margine esterno
            bindingoffset=0.5cm % Spazio extra per la rilegatura fisica
            ]{geometry}
\usepackage{pgfplotstable}
\usepackage{booktabs}
\usepackage{siunitx}
\usepackage{multicol}
\usepackage{supertabular}
\usepackage{caption}
\usepackage{longtable}
\usepackage[hidelinks]{hyperref}
\usepackage{parskip}
\usepackage{subcaption}
\pgfplotsset{compat=1.18}

\title{
    {\huge \bfseries Right Sizing Spike Server} \\[1em] % [1em] aggiunge spazio extra
    {\Large Relazione del progetto di PMCSN}
    {\\[2em] Studio delle }
}

\author{Luca Falasca \\ 
\href{mailto:lucafalasca08@gmail.com}{\texttt{lucafalasca08@gmail.com}} \\
\href{mailto:luca.falasca@alumni.uniroma2.eu}{\texttt{luca.falasca@alumni.uniroma2.eu}} \\[1em]
Matricola: 0334722}
\date{AA 2025/2026}

\begin{document}

\begin{titlepage}
    \centering % Centra tutto
    
    % --- Intestazione (Università/Dipartimento) ---
    {\scshape\LARGE Università di Roma Tor Vergata \par} % O la tua università
    \vspace{1cm}
    
    % --- Logo (Opzionale - scommenta se hai l'immagine) ---
    % \includegraphics[width=0.4\textwidth]{logo_universita.png}
    % \vspace{1.5cm}
    
    % --- Titolo e Sottotitolo ---
    {\huge\bfseries Right Sizing Spike Server \par}
    \vspace{0.5cm}
    {\Large Relazione del progetto di PMCSN \par}
    \vspace{2cm}
    
    % --- Autore/i ---
    {\Large
    {Luca Falasca \\ 
    \href{mailto:lucafalasca08@gmail.com}{\texttt{lucafalasca08@gmail.com}} \\
    \href{mailto:luca.falasca@alumni.uniroma2.eu}{\texttt{luca.falasca@alumni.uniroma2.eu}} \\[2em]
    Matricola: 0334722}
    }
    
    \vfill % Spinge tutto quello che segue verso il fondo pagina
    
    % --- Data / Anno Accademico ---
    {\large A.A 2025/2026}
    
\end{titlepage}

\tableofcontents
\clearpage
\section{Caso di Studio}
Il sistema oggetto di studio è un'architettura di data center per un Internet Service Provider, progettata per gestire dinamicamente le fluttuazioni di carico e garantire la Quality of Service (QoS) ottimizzando al contempo l'uso delle risorse.

Il problema principale affrontato è il "right-sizing", ovvero come evitare sia il sovradimensionamento (spreco di risorse) sia il sottodimensionamento (violazione degli SLA e degrado delle prestazioni), specialmente in presenza di fluttuazioni di carico a breve e lungo termine.

L'architettura proposta, come descritto nel caso di studio 6.2 del libro di testo “Performance Engineer”, si basa su un livello di scaling verticale che gestisce i picchi di carico improvvisi e di breve durata. Questo livello introduce uno Spike Server dedicato. Un Load Controller monitora un indicatore di picco (Spike Indicator, SI), definito come il numero di richieste concorrenti in esecuzione su un Web Server.

Il comportamento del sistema seguirebbe quanto descritto:
\begin{itemize}
    \item Quando l'indicatore SI supera una soglia di allarme SI\_max, le nuove richieste in arrivo non vengono più inviate al Web Server congestionato, ma vengono reindirizzate allo Spike Server.
    \item Quando il carico sul Web Server diminuisce e SI scende al di sotto della soglia, il routing delle richieste torna alla normalità.
\end{itemize}

\section{Obiettivi dello studio}
Lo studio si pone l'obiettivo di analizzare e validare l'efficacia del modello di autoscaling gerarchico attraverso la simulazione. Gli obiettivi specifici sono:
\begin{itemize}
    \item identificare il valore di carico di lavoro che supera i tempi di risposta richiesti
    \item Identificare il valore ottimale di SI\_max: Per un dato carico di lavoro (es. 40,000 richieste/ora), determinare il valore di SI\_max che minimizza il tempo di risposta medio del sistema o lo mantiene al di sotto di un valore target definito da un Service Level Agreement, come gli 8 secondi utilizzati nel libro.
    \item Studiare il comportamento del sistema sotto carichi crescenti: Analizzare le prestazioni del sistema (in particolare il tempo di risposta) al variare del tasso di arrivo delle richieste (carico leggero, medio, pesante) utilizzando la soglia SI\_max.
    \item Studiare il sistema con fluttuazioni a breve e a lungo termine: Analizzare le prestazioni del sistema introducendo fluttuazioni a breve e lungo termine, utilizzando sempre la soglia SI\_max
    \item Trovare due nuovi SI\_max per fluttuazioni a breve e a lungo termine per rimanere sotto un certo tempo di risposta (8 secondi)
\end{itemize}
\section{Modello concettuale}
Il modello descritto può essere schematizzato come in figura \ref{fig:modello_schematizzato}.
\begin{figure}[H]
    \centering
    \includegraphics[width=\textwidth]{images/modello_schematizzato.png}
    \caption{Modello concettuale del sistema di autoscaling gerarchico con Spike Server}
    \label{fig:modello_schematizzato}
\end{figure}
I job arrivano e a seconda del livello di pienezza dei webservers.
Nonostante il sistema possa sembrare troppo semplice per una analisi simulativa, ci sono una serie di aspetti che lo rendono complesso e difficilmente modellabile solo matematicamente senza semplificazioni:
\begin{itemize}
    \item Il routing non è probabilistico: il routing dei job non è semplicemente probabilistico ( 40\% su uno e 60\% su un altro), ma dipende strettamente dallo stato del webserver nel momento del routing. Questa complicazione rende molto difficile un'analisi statica, soprattutto nel transiente.
\end{itemize}

Spiegazione tempi di servizio esponenziale: Nel caso di studio affrontato nel libro viene utilizzata una distribuzione iperesponenziale nel tassi di servizio per modellare il fatto che ad un server arrivano job di dimensione molto variabile. Ovviamente questa cosa potrebbe essere modellata anche utilizzando delle size dei job differenti anziché agire sul tasso di servizio. Tuttavia ho deciso di attenermi al testo originale e utilizzare anche io dei tassi di servizio iperesponenziali per modellare questo comportamento. 

\subsection{Spike server}
Lo Spike Server avrà una potenza maggiore rispetto al webserver. In particolare avrà un tasso di servizio doppio rispetto a quello del webserver principale, in modo da poter smaltire rapidamente i picchi di carico. Per il calcolo di SI\_max è importante sottolineare che lo spike server è bene usarlo il meno possibile, in quanto è una risorsa più costosa rispetto al webserver. Quindi anziché cercare di minimizzare il tempo di risposta medio del sistema, si cerca di rispettare lo SLA (8 secondi) utilizzando il meno possibile lo spike server.

\subsection{Web Server}
Il web server sarà solamente uno, per semplicità di modellazione. Per questo motivo non vengono gestite le fluttuazioni a lungo termine, in quanto queste richiederebbero un sistema con più webserver per utilizzare una scaling orizzontale. 

\section{Modello delle specifiche}

Componenti fisici del sistema:
\begin{itemize}
    \item Web server principale
    \item Spike server
\end{itemize}
Componenti logici del sistema:
\begin{itemize}
    \item Load Controller
\end{itemize}

Specifiche del carico di lavoro:
\begin{itemize}
    \item Arrivi: Processo di arrivo iperesponenziale con cv = 4 e media 0.15 secondi (40,000 richieste/ora)
    \item Servizio: Distribuzione iperesponenziale con cv = 4 e media 0.16 secondi (web server), media 0.08 secondi (spike server)
    \item Scheduling: Processor Sharing
\end{itemize}


Classi di clienti:
\begin{itemize}
    \item Arrivi normali
    \item Token: sono gettoni di controllo che servono al Load Controller. C'è un numero fisso di token (SI\_max). Ogni volta che arriva una richiesta, se ci sono token disponibili, ne viene consumato uno e la richiesta viene inviata al webserver principale. Se non ci sono token disponibili, la richiesta viene inviata allo spike server. Quando una richiesta termina il servizio, il token viene restituito al Load Controller.
\end{itemize}


Logica di controllo:
\begin{itemize}
    \item Se \(SI < SI\_max\): invia al webserver principale
    \item Se \(SI \geq SI\_max\): invia allo spike server
\end{itemize}


Gestione della fine della simulazione:
Dato che il sistema viene studiato in un stato transitorio, esso fa riferimento ad un orizzonte temporale prefissato che va dai 120 secondi (per eliminare il bias della fase iniziale in cui le code sono ancora vuote) ai 1200 secondi di simulazione. Questo viene fatto nel report di riferimento e quindi per confrontare al meglio i risultati si è deciso di adottare lo stesso approccio. Inoltre, non verrà fatta allo scadere del tempo di simulazione una pulizia delle code, ma si considereranno solo i job completati entro il tempo di simulazione.

\section{Modello Computazionale}

\subsection{Stato del sistema}

\begin{itemize}
    \item SI: Spike indicator (numero di job nel web server)
    \item n\_spike: numero di job nello spike server
\end{itemize}

\subsection{Identificazione degli eventi}
\begin{itemize}
    \item arrivo di un nuovo job
    \item completamento di un job nel web server
    \item completamento di un job nello spike server
\end{itemize}

\subsection{Logica di controllo e routing}
Ogni volta che arriva un nuovo job:
\begin{itemize}
    \item Se \(SI \leq SI\_max\), viene incrementato SI di 1 e il job viene inviato al web server.
    \item Se \(SI > SI\_max\), viene incrementato n\_spike di 1 e il job viene inviato allo spike server.
\end{itemize} 

Ogni volta che job termina decremenenta il corrispettivo contatore restituendo il token.

\subsection{Implementazione dello scheduler}
Al contrario di quanto avviene nel caso di studio di riferimento in cui si usa uno scheduling processor sharing, in questo progetto userò uno scheduling FIFO per entrambi i server.


\subsection{Configurazione dei parametri delle distribuzioni}
\begin{itemize}
    \item Stream 0: Inter-arrivi (Iperesponenziale, media 0.15s, cv=4).
    \item Stream 1: Servizio Web Server (Iperesponenziale, media 0.16s, cv=4).
    \item Stream 2: Servizio Spike Server (Iperesponenziale, media 0.08s, cv=4).
\end{itemize}

    






\section{Verifica}
Ci sono varie verifiche che si possono fare andando a controllare se tutta una serie di caratteristiche volute si riflettano veramente nel modello simulativo costruito e quindi nei risultati ottenuti. In particolare, andrò ad effettuare le seguenti verifiche:
\begin{itemize}
    \item Verifica del corretto funzionamento di routing basato su soglia si \(SI_{max}\).
    \begin{itemize}
        \item Controllo che impostando \(SI_{max}=0\) venga utilizzato solo lo spike server.
        \item Controllo che impostando \(SI_{max}=\infty\) non venga mai utilizzato lo spike server.
    \end{itemize}
    \item Verifica delle metriche prestazionali:
    \begin{itemize}
        \item Verifico che aumentando \(SI_{max}\) l'utilizzazione del web server aumenta
        \item Verifico che il throughput sia pari al tasso di arrivo quando il sistema è stabile e uguale al tasso di servizio quando il sistema è sovraccarico.
    \end{itemize}
    % \item Verifica leggi operazionali
    % \begin{itemize}
    %     \item Verifica della legge di little: \(E[N] = \lambda E[R]\)
    % \end{itemize}
    % \item Verifica della distribuzione
    % \begin{itemize}
    %     \item Verifica che media e cv della distribuzione iperesponenziale corrispondano ai valori impostati.
    % \end{itemize}
\end{itemize}
    
\subsection{Verifica del routing basato su soglia \(SI_{max}\)}
\subsubsection{Verifica con \(SI_{max}=0\)}
Impostando \(SI_{max}=0\) si può verificare che tutti i job vengano instradati allo spike server. Infatti, come mostrato in figura \ref{fig:verifica_si_0}, l'utilizzazione del web server è nulla, mentre tutto il carico viene gestito dallo spike server.

\begin{figure}[H]
    \centering
    \includegraphics[width=0.8\textwidth]{images/verifica_si_0.png}
    \caption{Verifica routing con \(SI_{max}=0\)}
    \label{fig:verifica_si_0}
\end{figure}
\subsubsection{Verifica con \(SI_{max}=\infty\)}
Impostando \(SI_{max}=\infty\) si può verificare che nessun job venga instradato allo spike server. Infatti, come mostrato in figura \ref{fig:verifica_si_inf}, l'utilizzazione dello spike server è nulla, mentre tutto il carico viene gestito dal web server.

\begin{figure}[H]
    \centering
    \includegraphics[width=0.8\textwidth]{images/verifica_si_inf.png}
    \caption{Verifica routing con \(SI_{max}=\infty\)}
    \label{fig:verifica_si_inf}
\end{figure}

\subsection{Verifica delle metriche prestazionali}
\subsubsection{Verifica dell'utilizzazione al variare di \(SI_{max}\)}
Impostando un carico di lavoro fisso e variando il valore di \(SI_{max}\), si può verificare che l'utilizzazione del web server aumenti al crescere di \(SI_{max}\), come mostrato in figura \ref{fig:verifica_utilizzazione_si}.

\begin{figure}[H]
    \centering
    \includegraphics[width=0.8\textwidth]{images/verifica_utilizzazione_si.png}
    \caption{Verifica dell'utilizzazione del web server al variare di \(SI_{max}\)}
    \label{fig:verifica_utilizzazione_si}
\end{figure}

\begin{table}[H]
  \begin{center}
    \caption{Verifica dell'utilizzazione del web server al variare di \(SI_{max}\)}
    \pgfplotstabletypeset[
      multicolumn names, % allows to have multicolumn names
      col sep=comma, % the seperator in our .csv file
      display columns/0/.style={
        column name={$SI_{max}$},
        column type={S[round-mode=places, round-precision=0]},string type},  % use siunitx for formatting
      display columns/1/.style={
        column name={$\text{U}_{web}$},
        column type={S[round-mode=places, round-precision=4]},string type},
      display columns/2/.style={
        column name={$\text{U}_{web}$ CI95},
        column type={S[round-mode=places, round-precision=4]},string type},
      every head row/.style={
        before row={\toprule}, % have a rule at top
        after row={\midrule}, % rule under units
            },
        every last row/.style={after row=\bottomrule}, % rule at bottom
    ]{data_tables/graph_data_si_max.csv} % filename/path to file
  \end{center}
\end{table}

\subsubsection{Verifica del throughput}
Variando il carico di lavoro da 1 req/s a 12 req/s e impostando \(SI_{max} = \infty\), in modo da misurare solo il web server, si può verificare che il throughput del sistema non superi mai il suo limite teorico, che nel caso considerato è uguale a \(\min(\mu_{web}, \lambda_{web})\). Tuttavia, come evidenziato in Figura 6, il throughput misurato \(X_{measured} = N_{completed} / T_{sim}\) supera il limite teorico in condizioni di sovraccarico \((\rho>1)\). L'analisi di questa anomalia ha identificato la causa nella gestione della fine della simulazione (stopping condition).

Il protocollo di simulazione adottato, per mantenere stretta aderenza al caso di studio di riferimento, interrompe la simulazione a  \(T_{stop}=1200s\) senza attendere lo svuotamento delle code. Questa scelta metodologica introduce un bias sistematico quando il sistema è instabile:
\begin{itemize}
  \item Il calcolo del throughput considera solo i job completati entro $T_{stop}$.
  \item A causa della distribuzione iperesponenziale dei servizi (cv=4), i job con richieste di servizio brevi tendono a completare rapidamente, mentre i job con richieste lunghe ("heavy tail") tendono ad accumularsi in coda e rimanere incompleti allo scadere del tempo.
\end{itemize}

  Di conseguenza, il rapporto tra job completati e tempo trascorso risulta sovrastimato rispetto alla capacità reale del server di smaltire lavoro medio.



\begin{figure}[H]
    \centering
    \includegraphics[width=0.8\textwidth]{images/wrong_throughput.png}
    \caption{Throughput misurato che supera il valore teorico}
    \label{fig:wrong_throughput}
\end{figure}

Sebbene si sia scelto di mantenere questo approccio per garantire la comparabilità dei risultati con il report originale, è fondamentale notare che il simulatore si comporta correttamente se si considera la relazione fondamentale:

\[ X_{calculated} = U_{web} \cdot \mu_{web} \]

Come mostrato in Figura \ref{fig:correct_throughput} e nella Tabella \ref{tab:verifica_throughput2}, il throughput calcolato a partire dall'utilizzazione ($X_{\text{calculated}}$) rispetta perfettamente il limite teorico, confermando che la logica interna del simulatore è corretta e che la discrepanza in Figura 6 è un artefatto della metrica di misurazione scelta dallo studio di riferimento, non un errore implementativo.

\begin{figure}[H]
    \centering
    \includegraphics[width=0.8\textwidth]{images/correct_throughput.png}
    \caption{Throughput calcolato a partire dall'utilizzazione}
    \label{fig:correct_throughput}
\end{figure}

\begin{table}[H]
  \begin{center}
    \caption{Verifica del Throughput}
    \label{tab:verifica_throughput2}
    \pgfplotstabletypeset[
      multicolumn names, % allows to have multicolumn names
      col sep=comma, % the seperator in our .csv file
      display columns/0/.style={
        column name={$\lambda$},
        column type={S[round-mode=places, round-precision=0]},string type},  % use siunitx for formatting
      display columns/1/.style={
        column name={$X_{measured}$},
        column type={S[round-mode=places, round-precision=4]},string type},
      display columns/2/.style={
        column name={$X_{measured}$ CI95},
        column type={S[round-mode=places, round-precision=4]},string type},
      display columns/3/.style={
        column name={$X_{theoretical}$},
        column type={S[round-mode=places, round-precision=4]},string type
        },
      display columns/4/.style={
        column name={$X_{calculated}$},
        column type={S[round-mode=places, round-precision=4]},string type
        },
      display columns/5/.style={
        column name={$X_{calculated}$ CI95},
        column type={S[round-mode=places, round-precision=4]},string type
        },
      every head row/.style={
        column name={},
        before row={\toprule}, % have a rule at top
        after row={\midrule}, % rule under units
            },
        every last row/.style={after row=\bottomrule}, % rule at bottom
    ]{data_tables/data_throughput_si_max_inf.csv} % filename/path to file
  \end{center}
\end{table}

\section{Validazione}
Dato che in questo studio non sono disponibili dati reali con cui confrontare i risultati ottenuti dalla simulazione, la validazione del modello simulativo confronterà i risultati ottenuti con quelli ottenuti dallo studio di riferimento nella prossima sezione del report.


\subsection{Confronto tempi di risposta}
Nello studio di riferimento vengono effettuati dei test sui tempi di risposta sul webserver e sullo spike server. 

\begin{figure}[H]
    \centering
    \includegraphics[width=1\textwidth]{images/response_time_serazzi.png}
    \caption{Tempi di risposta del webserver e dello spike server in funzione del carico nel caso di studio di Serazzi}
    \label{fig:response_time_serazzi}
\end{figure}

\begin{figure}[H]
    \centering
    \includegraphics[width=1\textwidth]{images/response_time_web_vs_si_max.png}
    \caption{Tempi di risposta del webserver in funzione del SI\_max ottenuti dalla simulazione.}
    \label{fig:response_time_web_sim}
\end{figure}

\begin{figure}[H]
    \centering
    \includegraphics[width=1\textwidth]{images/response_time_spike_vs_si_max.png}
    \caption{Tempi di risposta dello spike server in funzione del SI\_max ottenuti dalla simulazione.}
    \label{fig:response_time_spike_sim}
\end{figure}

\begin{table}[H]
  \begin{center}
    \caption{Validazione tempi di risposta del webserver}
    \label{tab:validazione_response_time_web_vs_si}
    \pgfplotstabletypeset[
      multicolumn names, % allows to have multicolumn names
      col sep=comma, % the seperator in our .csv file
      display columns/0/.style={
        column name={$SI_{max}$},
        column type={S[round-mode=places, round-precision=0]},string type},  % use siunitx for formatting
      display columns/1/.style={
        column name={$E[T_s]_{web}$},
        column type={S[round-mode=places, round-precision=4]},string type},
      display columns/2/.style={
        column name={$E[T_s]_{web}$ CI95},
        column type={S[round-mode=places, round-precision=4]},string type},
      every head row/.style={
        column name={},
        before row={\toprule}, % have a rule at top
        after row={\midrule}, % rule under units
            },
        every last row/.style={after row=\bottomrule}, % rule at bottom
    ]{data_tables/response_time_web_vs_si_max.csv} % filename/path to file
  \end{center}
\end{table}

\begin{table}[H]
  \begin{center}
    \caption{Validazione tempi di risposta dello spike server}
    \label{tab:validazione_response_time_spike_vs_si_max}
    \pgfplotstabletypeset[
      multicolumn names, % allows to have multicolumn names
      col sep=comma, % the seperator in our .csv file
      display columns/0/.style={
        column name={$SI_{max}$},
        column type={S[round-mode=places, round-precision=0]},string type},  % use siunitx for formatting
      display columns/1/.style={
        column name={$E[T_s]_{spike}$},
        column type={S[round-mode=places, round-precision=4]},string type},
      display columns/2/.style={
        column name={$E[T_s]_{spike}$ CI95},
        column type={S[round-mode=places, round-precision=4]},string type},
      every head row/.style={
        column name={},
        before row={\toprule}, % have a rule at top
        after row={\midrule}, % rule under units
            },
        every last row/.style={after row=\bottomrule}, % rule at bottom
    ]{data_tables/response_time_spike_vs_si_max.csv} % filename/path to file
  \end{center}
\end{table}

Controntandoli con i risultati ottenuti dalla simulazione progettata, si nota come l'andamento sia essenzialmente lo stesso, tuttavia i valori dei tempi differiscono. Questo è dovuto dal tempo di simulazione che il caso di studio originale non esplicita direttamente e quindi probabilmente differente dai 5000 secondi utilizzati nelle simulazioni. Tuttavia per non rendere troppo lunghi gli esperimenti, è stata eseguita una simulazione con \(SI{max}\) = 160 e aumentato il tempo di simulazione a 20000 secondi, per dimostrare che il motivo della differenza nei tempi di risposta è dovuto proprio al tempo di simulazione. 

\begin{figure}[H]
    \centering
    \includegraphics[width=1\textwidth]{images/run_20000sec.png}
    \caption{Tempi di risposta del webserver con \(SI_{max} = 160\) ottenuti dalla simulazione con 20000 secondi di simulazione.}
    \label{fig:response_time_web_sim_20000}
\end{figure}

\subsection{Confronto Throughput}
Nello studio di riferimento vengono effettuati dei test sul throughput del webserver.

\begin{figure}[H]
    \centering
    \includegraphics[width=1\textwidth]{images/throughtput_web_spike_serazzi.png}
    \caption{Throughput del webserver e dello spike server in funzione del carico nel caso di studio di Serazzi}
    \label{fig:throughput_serazzi}
\end{figure}

\begin{figure}[H]
    \centering
    \includegraphics[width=1\textwidth]{images/throughput_web_vs_si_max.png}
    \caption{Throughput del webserver in funzione del SI\_max ottenuti dalla simulazione.}
    \label{fig:throughput_web_sim}
\end{figure}

\begin{figure}[H]
    \centering
    \includegraphics[width=1\textwidth]{images/throughput_spike_vs_si_max.png}
    \caption{Throughput dello spike server in funzione del SI\_max ottenuti dalla simulazione.}
    \label{fig:throughput_spike_sim}
\end{figure}

L'andamento e i valori qui sono essenzialmente gli stessi. Purtroppo il caso di studio originale non fornisce i valori numerici precisi su cui ha costruito i grafici per confrontarli direttamente.




\section{Risultati}
Gli esperimenti condotti mirano a raggiungere gli obiettivi delineati nella sezione 2.
\subsection{Obiettivo 1}
Come si può evincere dalla figura \ref{fig:response_time_vs_si_max}, il tempo di risposta medio \(E[R]\) rimane al di sotto della soglia di 8 secondi per valori di \(SI_{max}\) fino a circa 90. Esattamente come suggerito nello studio di riferimento, il valore ottimale di \(SI_{max}\) che massimizza l'utilizzo del web server mantenendo il rispetto dello SLA si attesta intorno a 80-90. Considerando che con l'intervallo di confidenza con 90 si sfora, il valore più prudente da adottare per \(SI_{max}\) tra i due risulta essere 80.
\begin{figure}[H]
    \centering
    \includegraphics[width=1\textwidth]{images/response_time_vs_si_max.png}
    \caption{Tempo di risposta medio in funzione di \(SI_{max}\)}
    \label{fig:response_time_vs_si_max}
\end{figure}

\begin{table}[H]
    \begin{center}
        \caption{Tempo di risposta medio in funzione di \(SI_{max}\)}
        \pgfplotstabletypeset[
        multicolumn names, % allows to have multicolumn names
        col sep=comma, % the seperator in our .csv file
        display columns/0/.style={
            column name={$SI_{max}$},
            column type={S[round-mode=places, round-precision=0]},string type
        },
        display columns/1/.style={
            column name={$E[T_s]$},
            column type={S[round-mode=places, round-precision=4]},string type
        },
        display columns/2/.style={
            column name={$E[T_s]$ CI95},
            column type={S[round-mode=places, round-precision=4]},string type
        },
        every head row/.style={
            before row=\toprule, % adds a top rule before the table header
            after row=\midrule,  % adds a mid rule after the table header
        },
        every last row/.style={
            after row=\bottomrule, % adds a bottom rule after the table
        },
        ]{../docs/data_tables/response_time_vs_si_max.csv}
    \end{center}
\end{table}
 

\subsection{Obiettivo 2}
Come mostrato in figura \ref{fig:response_time_vs_arrival_rate_vs_si_max}, accade lo stesso fenomeno descritto nello studio di riferimento: al crescere del tasso di arrivo, il tempo di risposta medio aumenta all'aumentare del tasso degli arrivi, fino a che non si arriva a saturare il sistema (\(\rho > 1\)) e quindi a superare la soglia \(SI_{max}\). A questo punto, il tempo di risposta comincia a calare fino a che anche lo spike server si satura e il tempo di risposta ricomincia a salire per valori di \(\lambda \geq 11\) nella maggior parte dei casi. Per valori molto bassi di \(SI_{max}\) il tempo di risposta continua a salire quasi sempre, questo accade perché essenzialmente è come se il webserver non venisse utilizzato e tutto il carico finisse sullo spike server, che quindi tende a saturarsi già per valori più bassi di \(\lambda\). Nelle tabelle sottostanti (Tabella \ref{tab:response_time_vs_arrival_rate_vs_si_max} e Tabella \ref{tab:response_time_vs_arrival_rate_vs_si_max_enhanced_spike}) sono riportati i valori numerici però mostrando solo i valori di \(SI_{max}\) a multipli di 20, questa scelta è dovuta alla grande quantità di dati generata. In ogni caso i dati sono consultabili nei file csv all'indirizzo \url{https://github.com/LucaFalasca/RightSizing-SpikeServer}.

\begin{figure}[H]
    \centering
    \includegraphics[width=1\textwidth]{images/response_time_vs_arrival_rate_vs_si_max.png}
    \caption{Tempo di risposta medio in funzione del tasso di arrivo e di \(SI_{max}\)}
    \label{fig:response_time_vs_arrival_rate_vs_si_max}
\end{figure}

\begin{table}[H]
    \begin{center}
        \caption{Tempo di risposta medio in funzione del tasso di arrivo e di \(SI_{max}\)}
        \label{tab:response_time_vs_arrival_rate_vs_si_max}
        \begin{minipage}[t]{0.48\textwidth}
            \pgfplotstabletypeset[
            multicolumn names, % allows to have multicolumn names
            col sep=comma, % the seperator in our .csv file
            display columns/0/.style={
                column name={SI max},
                column type={S[round-mode=places, round-precision=0]},string type
            },
            display columns/1/.style={
                column name={\(\lambda\) },
                column type={S[round-mode=places, round-precision=0]},string type
            },
            display columns/2/.style={
                column name={$E[T_s]$},
                column type={S[round-mode=places, round-precision=4]},string type
            },
            display columns/3/.style={
                column name={$E[T_s]$ CI95},
                column type={S[round-mode=places, round-precision=4]},string type
            },
            every head row/.style={
                before row=\toprule, % adds a top rule before the table header
                after row=\midrule,  % adds a mid rule after the table header
            },
            every last row/.style={
                after row=\bottomrule, % adds a bottom rule after the table
            },
            ]{../docs/data_tables/response_time_vs_arrival_rate_vs_si_max_1.csv}
        \end{minipage}
        \begin{minipage}[t]{0.48\textwidth}
            \pgfplotstabletypeset[
            multicolumn names, % allows to have multicolumn names
            col sep=comma, % the seperator in our .csv file
            display columns/0/.style={
                column name={SI max},
                column type={S[round-mode=places, round-precision=0]},string type
            },
            display columns/1/.style={
                column name={\(\lambda\) },
                column type={S[round-mode=places, round-precision=0]},string type
            },
            display columns/2/.style={
                column name={$E[T_s]$},
                column type={S[round-mode=places, round-precision=4]},string type
            },
            display columns/3/.style={
                column name={$E[T_s]$ CI95},
                column type={S[round-mode=places, round-precision=4]},string type
            },
            every head row/.style={
                before row=\toprule, % adds a top rule before the table header
                after row=\midrule,  % adds a mid rule after the table header
            },
            every last row/.style={
                after row=\bottomrule, % adds a bottom rule after the table
            },
            ]{../docs/data_tables/response_time_vs_arrival_rate_vs_si_max_2.csv}
        \end{minipage}
        
        
    \end{center}
\end{table}

\subsection{Obiettivo 3}
Nell'obiettivo 3 si va ad duplicare la potenza dello spike server, per verificare se questo evita il saturamento dello spike server e quindi il successivo aumento del tempo di risposta medio. Come mostrato in figura \ref{fig:response_time_vs_arrival_rate_vs_si_max_enhanced_spike}, lo spike server non arriva più a saturarsi e il tempo di risposta continua a diminuire all'aumentare del tasso di arrivo. I dati tabulari sono riportati nella Tabella \ref{tab:response_time_vs_arrival_rate_vs_si_max_enhanced_spike}.


\begin{figure}[H]
    \centering
    \includegraphics[width=1\textwidth]{images/response_time_vs_arrival_rate_vs_si_max_enhanced_spike.png}
    \caption{Tempo di risposta medio in funzione del tasso di arrivo e di \(SI_{max}\) con spike server potenziato}
    \label{fig:response_time_vs_arrival_rate_vs_si_max_enhanced_spike}
\end{figure}

\begin{table}
    \begin{center}
        \caption{Tempo di risposta medio in funzione del tasso di arrivo e di \(SI_{max}\) con spike server potenziato}
        \label{tab:response_time_vs_arrival_rate_vs_si_max_enhanced_spike}
        \begin{minipage}[t]{0.48\textwidth}
            \pgfplotstabletypeset[
            multicolumn names, % allows to have multicolumn names
            col sep=comma, % the seperator in our .csv file
            display columns/0/.style={
                column name={SI max},
                column type={S[round-mode=places, round-precision=0]},string type
            },
            display columns/1/.style={
                column name={\(\lambda\) },
                column type={S[round-mode=places, round-precision=0]},string type
            },
            display columns/2/.style={
                column name={$E[T_s]$},
                column type={S[round-mode=places, round-precision=4]},string type
            },
            display columns/3/.style={
                column name={$E[T_s]$ CI95},
                column type={S[round-mode=places, round-precision=4]},string type
            },
            every head row/.style={
                before row=\toprule, % adds a top rule before the table header
                after row=\midrule,  % adds a mid rule after the table header
            },
            every last row/.style={
                after row=\bottomrule, % adds a bottom rule after the table
            },
            ]{../docs/data_tables/response_time_vs_arrival_rate_vs_si_max_enhanced_spike_1.csv}
        \end{minipage}
        \begin{minipage}[t]{0.48\textwidth}
            \pgfplotstabletypeset[
            multicolumn names, % allows to have multicolumn names
            col sep=comma, % the seperator in our .csv file
            display columns/0/.style={
                column name={SI max},
                column type={S[round-mode=places, round-precision=0]},string type
            },
            display columns/1/.style={
                column name={\(\lambda\) },
                column type={S[round-mode=places, round-precision=0]},string type
            },
            display columns/2/.style={
                column name={$E[T_s]$},
                column type={S[round-mode=places, round-precision=4]},string type
            },
            display columns/3/.style={
                column name={$E[T_s]$ CI95},
                column type={S[round-mode=places, round-precision=4]},string type
            },
            every head row/.style={
                before row=\toprule, % adds a top rule before the table header
                after row=\midrule,  % adds a mid rule after the table header
            },
            every last row/.style={
                after row=\bottomrule, % adds a bottom rule after the table
            },
            ]{../docs/data_tables/response_time_vs_arrival_rate_vs_si_max_enhanced_spike_2.csv}
        \end{minipage}
    \end{center}
\end{table}

\section{Conclusioni}
Da questo lavoro si ottiene conferma di quanto già evidenziato nello studio di Serazzi, il valore ottimale per \(SI_{max}\) si attesta intorno a 80-90 per il carico di lavoro di 6.66 req/s, che massimizza l'utilizzo del web server mantenendo il rispetto dello SLA sul tempo di risposta medio. Inoltre lo studio evidenzia che la capacità dello spike server va tarata in base al carico di lavoro atteso, perché altrimenti si rischia di saturare anche lui e vanificare l'efficacia del sistema.
\clearpage
\appendix
\section{Analisi Transitorio}
L'analisi del transitorio è stata effettuata per osservare l'andamento del tempo di risposta nel tempo, in modo da analizzare dopo quanto tempo il sistema raggiunge lo stato stazionario. Per fare ciò come negli esperimenti precedenti è stato utilizzato un unico seed = 8, un tempo di simulazione di 5000 secondi ma stavolta ovviamente senza eliminare la fase transitoria, che è proprio quella che ci interessa studiare. Il campionamento del tempo di risposta è stato effettuato ogni 100 secondi.

\subsection{Transitorio Obiettivo 1}
Qui si mostra l'analisi del transitorio nel caso dell'Obiettivo 1, quindi con \(\lambda =  6.66 req/s\) e \(SI_{max} = 80\).
Si vede che il tempo di risposta tende a stabilizzarsi intorno al valore di 1500 secondi, anche se continua a salire lentamente. Inoltre come si vede dai valori tabellari, l'intervallo di confidenza si stringe sempre di più, segno che il sistema sta raggiungendo uno stato stazionario.

\begin{figure}[H]
    \centering
    \includegraphics[width=1\textwidth]{images/transient_response_time_obj1.png}
    \caption{Andamento del tempo di risposta nel tempo per l'Obiettivo 1.}
    \label{fig:transient_response_time_obj1}
\end{figure}    

\begin{table}[H]
    \begin{center}
        \caption{Tempo di risposta medio in funzione del tempo per \(SI_{max} = 80\) e \(\lambda = 6.66 req/s\).}
        \pgfplotstabletypeset[
        multicolumn names, % allows to have multicolumn names
        col sep=comma, % the seperator in our .csv file
        display columns/0/.style={
            column name={$t$},
            column type={S[round-mode=places, round-precision=0]},string type
        },
        display columns/1/.style={
            column name={$E[T_s]$},
            column type={S[round-mode=places, round-precision=4]},string type
        },
        display columns/2/.style={
            column name={$E[T_s]$ CI95},
            column type={S[round-mode=places, round-precision=4]},string type
        },
        every head row/.style={
            before row=\toprule, % adds a top rule before the table header
            after row=\midrule,  % adds a mid rule after the table header
        },
        every last row/.style={
            after row=\bottomrule, % adds a bottom rule after the table
        },
        ]{../docs/data_tables/transient_response_time_obj1.csv}
    \end{center}
\end{table}

\subsection{Obiettivo 2}
Sono state eseguite le analisi del transitorio per tutte le combinazioni di \(SI_{max}\) da 10 a 160 e tassi di arrivo da 1 req/s a 12 req/s. Qui esporrò solo le combinazioni più significative e rappresentative. In ogni caso, anche se sul report ne esporrò solo un sottoinsieme, tutte le immagini e tutte le tabelle con i dati sono disponibili nella cartella \texttt{docs/images} e \texttt{docs/data\_tables} del progetto.

Dagli esperimenti si possono fare una serie di considerazioni:
\begin{itemize}
    \item Per i casi con arrival rate basso e alto la convergenza sembra arrivare molto velocemente (Figure \ref{fig:transient_response_time_arrival_rate_2}, \ref{fig:transient_response_time_arrival_rate_4}, \ref{fig:transient_response_time_arrival_rate_10}, \ref{fig:transient_response_time_arrival_rate_12}). Questo perché nel primo caso il sistema è poco carico e quindi riesce a smaltire velocemente le richieste, mentre nel secondo caso il sistema è sovraccarico e quindi il tempo di risposta cresce rapidamente fino a stabilizzarsi su un valore alto.
    \item I casi più interessanti sono quelli con arrival rate medio vicino alla saturazione del sistema (Figure \ref{fig:transient_response_time_arrival_rate_6}, \ref{fig:transient_response_time_arrival_rate_7}). In questi casi, in cui il sistema è carico ma con una utilizzazione tale per cui il web server è quasi sempre pieno e lo spike server viene attivato e disattivato frequentemente, il tempo di risposta impiega più tempo a stabilizzarsi, soprattutto per valori di \(SI_{max}\) alti. Questo comportamento è osservabile anche nelle figure \ref{fig:transient_response_time_si_max_140}, \ref{fig:transient_response_time_si_max_160} che mostrano lo stesso fenomento da un'altra prospettiva.
\end{itemize}

\subsection{Obiettivo 3}
Per l'obiettivo 3 non sono state effettuate analisi del transitorio, in quanto ci si aspetta lo stesso comportamento dell'obiettivo 2, dato che l'unica differenza è la potenza dello spike server.

\begin{figure}[H]
    \centering
    \begin{subfigure}[b]{0.49\textwidth}
        \includegraphics[width=\textwidth]{images/transient_response_time_arrival_rate_2.png}    
        \caption{\( \lambda = 2 \)}
        \label{fig:transient_response_time_arrival_rate_2}
    \end{subfigure}
    \begin{subfigure}[b]{0.49\textwidth}
        \includegraphics[width=\textwidth]{images/transient_response_time_arrival_rate_4.png}    
        \caption{\( \lambda = 4 \)}
        \label{fig:transient_response_time_arrival_rate_4}
    \end{subfigure}

    \vspace{0.5cm}
    \begin{subfigure}[b]{0.49\textwidth}
        \includegraphics[width=\textwidth]{images/transient_response_time_arrival_rate_6.png}    
        \caption{\( \lambda = 6 \)}   
        \label{fig:transient_response_time_arrival_rate_6}
    \end{subfigure}
    \begin{subfigure}[b]{0.49\textwidth}
        \includegraphics[width=\textwidth]{images/transient_response_time_arrival_rate_7.png}    
        \caption{\( \lambda = 7 \)}   
        \label{fig:transient_response_time_arrival_rate_7}
    \end{subfigure}

    \vspace{0.5cm}
    \begin{subfigure}[b]{0.49\textwidth}
        \includegraphics[width=\textwidth]{images/transient_response_time_arrival_rate_10.png}    
        \caption{\( \lambda = 10 \)}   
        \label{fig:transient_response_time_arrival_rate_10}
    \end{subfigure}
    \begin{subfigure}[b]{0.49\textwidth}
        \includegraphics[width=\textwidth]{images/transient_response_time_arrival_rate_12.png}    
        \caption{\( \lambda = 12 \)}   
        \label{fig:transient_response_time_arrival_rate_12}
    \end{subfigure}
    \caption{Andamento del tempo di risposta nel tempo per diversi tassi di arrivo \(\lambda\).}
    \label{fig:transient_response_time_various_arrival_rate}
\end{figure}

\begin{figure}[H]
    \centering
    \begin{subfigure}[b]{0.49\textwidth}
        \includegraphics[width=\textwidth]{images/transient_response_time_si_max_10.png}    
        \caption{\(SI_{max} = 10\)}
        \label{fig:transient_response_time_si_max_10}
    \end{subfigure}
    \begin{subfigure}[b]{0.49\textwidth}
        \includegraphics[width=\textwidth]{images/transient_response_time_si_max_30.png}    
        \caption{\(SI_{max} = 30\)}
        \label{fig:transient_response_time_si_max_30}
    \end{subfigure}

    \vspace{0.5cm}
    \begin{subfigure}[b]{0.49\textwidth}
        \includegraphics[width=\textwidth]{images/transient_response_time_si_max_80.png}    
        \caption{\(SI_{max} = 80\)}   
        \label{fig:transient_response_time_si_max_80}
    \end{subfigure}
    \begin{subfigure}[b]{0.49\textwidth}
        \includegraphics[width=\textwidth]{images/transient_response_time_si_max_90.png}    
        \caption{\(SI_{max} = 90\)}   
        \label{fig:transient_response_time_si_max_90}
    \end{subfigure}
    \label{fig:transient_response_time_various_si_max}

    \vspace{0.5cm}
    \begin{subfigure}[b]{0.49\textwidth}
        \includegraphics[width=\textwidth]{images/transient_response_time_si_max_140.png}    
        \caption{\(SI_{max} = 140\)}   
        \label{fig:transient_response_time_si_max_140}
    \end{subfigure}
    \begin{subfigure}[b]{0.49\textwidth}
        \includegraphics[width=\textwidth]{images/transient_response_time_si_max_160.png}    
        \caption{\(SI_{max} = 160\)}   
        \label{fig:transient_response_time_si_max_160}
    \end{subfigure}
    \caption{Andamento del tempo di risposta nel tempo per diverse soglie di \(SI_{max}\).}
    \label{fig:transient_response_time_various_si_max}
\end{figure}

\begin{table}[H]
    \begin{center}
        \caption{Tempo di risposta medio in funzione del tempo per alcune combinazioni di \(SI_{max}\) e \(\lambda\).}
        \label{tab:response_time_vs_arrival_rate_vs_si_max_enhanced_spike}
        \begin{minipage}[t]{0.42\textwidth}
            \pgfplotstabletypeset[
            multicolumn names, % allows to have multicolumn names
            col sep=comma, % the seperator in our .csv file
            display columns/0/.style={
                column name={\(t\)},
                column type={S[round-mode=places, round-precision=0]},string type
            },
            display columns/1/.style={
                column name={$E[T_s]$},
                column type={S[round-mode=places, round-precision=4]},string type
            },
            display columns/2/.style={
                column name={$E[T_s]$ CI95},
                column type={S[round-mode=places, round-precision=4]},string type
            },
            every head row/.style={
                before row={
                    \toprule
                    % Cella 1: Vuota (occupa 1 posto)
                    \multicolumn{1}{c}{} & 
                    % Cella 2: Titolo (occupa 2 posti, colonne 2 e 3)
                    \multicolumn{2}{c}{\textbf{\(SI_{max} = 10 \bigwedge \lambda = 2\)}} \\ 
                    % Linea che copre solo le colonne 2 e 3
                    \cmidrule(lr){1-3}
                }, % adds a top rule before the table header
                after row=\midrule,  % adds a mid rule after the table header
            },
            every last row/.style={
                after row=\bottomrule, % adds a bottom rule after the table
            },
            ]{../docs/data_tables/transient_response_time_si_max_10_arrival_rate_2.csv}
        \end{minipage}
        \begin{minipage}[t]{0.28\textwidth}
            \pgfplotstabletypeset[
            multicolumn names, % allows to have multicolumn names
            col sep=comma, % the seperator in our .csv file
            columns={Transient Response Time Mean, Transient Response Time CI95},
            display columns/0/.style={
                column name={$E[T_s]$},
                column type={S[round-mode=places, round-precision=4]},string type
            },
            display columns/1/.style={
                column name={$E[T_s]$ CI95},
                column type={S[round-mode=places, round-precision=4]},string type
            },
            every head row/.style={
                before row={
                    \toprule
                    \multicolumn{2}{c}{\textbf{\(SI_{max} = 130 \bigwedge \lambda = 8\)}} \\ 
                    \cmidrule(lr){1-2} % La linea copre dalla colonna 1 alla 2
                }, % adds a top rule before the table header
                after row=\midrule,  % adds a mid rule after the table header
            },
            every last row/.style={
                after row=\bottomrule, % adds a bottom rule after the table
            },
            ]{../docs/data_tables/transient_response_time_si_max_130_arrival_rate_8.csv}
        \end{minipage}
        \begin{minipage}[t]{0.28\textwidth}
            \pgfplotstabletypeset[
            multicolumn names, % allows to have multicolumn names
            col sep=comma, % the seperator in our .csv file
            columns={Transient Response Time Mean, Transient Response Time CI95},
            display columns/0/.style={
                column name={$E[T_s]$},
                column type={S[round-mode=places, round-precision=4]},string type
            },
            display columns/1/.style={
                column name={$E[T_s]$ CI95},
                column type={S[round-mode=places, round-precision=4]},string type
            },
            every head row/.style={
                before row={
                    \toprule
                    \multicolumn{2}{c}{\textbf{\(SI_{max} = 160 \bigwedge \lambda = 12\)}} \\ 
                    \cmidrule(lr){1-2} % La linea copre dalla colonna 1 alla 2
                }, % adds a top rule before the table header
                after row=\midrule,  % adds a mid rule after the table header
            },
            every last row/.style={
                after row=\bottomrule, % adds a bottom rule after the table
            },
            ]{../docs/data_tables/transient_response_time_si_max_160_arrival_rate_12.csv}
        \end{minipage}
    \end{center}
\end{table}
\end{document}
