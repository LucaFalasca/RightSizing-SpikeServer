\section{Obiettivi dello studio}
Lo studio si pone l'obiettivo di analizzare e validare l'efficacia del modello di autoscaling gerarchico attraverso la simulazione. Gli obiettivi specifici sono:
\begin{itemize}
    \item identificare il valore di carico di lavoro che supera i tempi di risposta richiesti
    \item Identificare il valore ottimale di SI\_max: Per un dato carico di lavoro (es. 40,000 richieste/ora), determinare il valore di SI\_max che minimizza il tempo di risposta medio del sistema o lo mantiene al di sotto di un valore target definito da un Service Level Agreement, come gli 8 secondi utilizzati nel libro.
    \item Studiare il comportamento del sistema sotto carichi crescenti: Analizzare le prestazioni del sistema (in particolare il tempo di risposta) al variare del tasso di arrivo delle richieste (carico leggero, medio, pesante) utilizzando la soglia SI\_max.
    \item Studiare il sistema con fluttuazioni a breve termine: Analizzare le prestazioni del sistema introducendo fluttuazioni a breve termine, utilizzando sempre la soglia SI\_max
    \item Trovare due nuovi SI\_max per fluttuazioni a breve termine per rimanere sotto un certo tempo di risposta (8 secondi)
\end{itemize}





\subsection{Obiettivi 2}
Gli obiettivi specifici sono: 
\begin{itemize} 
    \item \textbf{Valutazione della capacità nominale}: Identificare il limite critico del carico di lavoro (tasso di arrivo) che porta alla violazione dei tempi di risposta definiti dagli SLA in assenza di meccanismi di scaling. 
    \item \textbf{Ottimizzazione della soglia SImax}: Determinare il valore ottimale dello \textit{Spike Indicator} (SI) — inteso come numero totale di job nel sistema (in coda e in servizio) — che minimizzi il tempo di risposta medio o lo mantenga entro il target di 8 secondi per un carico di lavoro di riferimento (es. 40.000 richieste/ora). 
    \item \textbf{Analisi di sensibilità sotto carichi crescenti}: Studiare le prestazioni del sistema al variare del tasso di arrivo per testare la stabilità della soglia SImax in scenari di carico leggero, medio e pesante. 
    \item \textbf{Analisi della robustezza ai picchi (\textit{Burstiness})}: Valutare l'efficacia dello \textit{Spike Server} (risorsa tipicamente più performante del Web Server principale) nel gestire fluttuazioni di carico a breve termine modellate tramite distribuzioni iperesponenziali. 
    \item \textbf{Ottimizzazione dinamica per scenari di picco}: Identificare valori specifici di SImax capaci di garantire il rispetto degli SLA anche in presenza di diverse intensità di picco improvviso, confrontando l'efficacia della soglia fissa rispetto a scenari di carico variabile. 
\end{itemize}

\subsection{Obiettivi 3}
Gli obiettivi specifici sono:
\begin{itemize}
    \item Determinare il valore di SI\_max più alto possibile (provando vari valori di SI) che mantenga comunque il tempo di risposta medio \[E[R] \leq 8\] secondi, con un tasso di arrivo iperesponenziale con media di 40,000 richieste/ora.
    \item Analizzare come varia il tempo di risposta medio al variare del coefficiente di variazione degli arrivi iperesponenziali (fluttuazioni a breve termine) mantenendo il tasso di arrivo medio a 40,000 richieste/ora e utilizzando il valore di SI\_max trovato nel punto precedente.
    \item Verificare come cambia il contesto se lo spike server ha un tasso di servizio triplo rispetto al webserver principale, invece che doppio come nel caso base.
\end{itemize}