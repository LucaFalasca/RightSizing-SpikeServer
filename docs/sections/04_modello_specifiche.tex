\section{Modello delle specifiche}
Queste variabili matematiche, sono la rappresentazione dello stato del sistema:
\begin{itemize}
    \item \(SI(t) \in [0, SI_{max})\): Spike indicator, indica il numero di job presenti nel web server al tempo t.
    \item \(n_{spike}(t) \in [0, \infty)\): numero di job presenti nello spike server al tempo t.
\end{itemize}

per cui lo stato del sistema è definito come la tupla \(S(t) = (SI(t), n_{spike}(t))\).


\subsection{Componenti fisici del sistema: }
\begin{itemize}
    \item Web server principale: server principale che gestisce le richieste in arrivo.
    \item Spike server: server secondario che viene attivato quando il carico sul web server principale supera una certa soglia (\(SI_{max}\)).
\end{itemize}
Componenti logici del sistema:
\begin{itemize}
    \item Load Controller: componente che monitora l'indicatore di picco (Spike Indicator, SI) e decide il routing delle richieste in base al valore di SI e alla soglia \(SI_{max}\).
\end{itemize}

\subsection{Specifiche del carico di lavoro:}
\begin{itemize}
    \item Arrivi: 
        \begin{itemize}
            \item Valore base: Processo di arrivo iperesponenziale con cv = 4 e media 0.15 secondi (6.66 req/s)
            \item Valore stress: Processo di arrivo iperesponenziale con carico variabile da 1 req/s a 12 req/s
        \end{itemize}
    \item Servizio: Distribuzione iperesponenziale con cv = 4 e media 0.16 secondi (web server e spike server), media 0.08 secondi (spike server nell'obiettivo 3)
    \item Scheduling: Processor Sharing
\end{itemize}
Vale la pena notare che, nel caso base del webserver, siccome \(\rho > 1\), il sistema è instabile e senza spike server il tempo di risposta divergerebbe a infinito.

\subsection{Logica di controllo:}
\begin{itemize}
    \item Arrivi nuovi job:
        \begin{itemize}
            \item Se \(SI < SI_{max}\): invia al webserver principale
            \item Se \(SI \geq SI_{max}\): invia allo spike server
        \end{itemize}
    \item Completamento job:
        \begin{itemize}
            \item Decrementa il contatore SI se il job era nel webserver principale
            \item Decrementa il contatore n\_spike se il job era nello spike server
        \end{itemize}
\end{itemize}
Nella logica originale del caso di studio, il valore di SI parte da \(SI_{max}\) e viene decrementato ad ogni completamento di un job. La logica da me adottata, nonostante sia inversa, è perfettamente equivalente.

\subsection{Metrica di valutazione delle prestazioni:}
Lo SLA da rispettare è un tempo di risposta medio \(E[R] \leq 8\) secondi.

\subsection{Gestione della fine della simulazione:}
Dato che il sistema viene studiato a regime, il sistema viene simulato per un tempo fisso di 5000 secondi, scartando i primi 500 secondi di fase transitoria. Inoltre una volta raggiunto il tempo di stop, vengono completati tutti i job ancora in servizio, fermando gli arrivi di nuovi job allo scadere del tempo di stop.