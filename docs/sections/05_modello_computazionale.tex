\section{Modello Computazionale}

\subsection{Stato del sistema}

\begin{itemize}
    \item SI: Spike indicator (numero di job nel web server)
    \item n\_spike: numero di job nello spike server
\end{itemize}

\subsection{Identificazione degli eventi}
\begin{itemize}
    \item arrivo di un nuovo job
    \item completamento di un job nel web server
    \item completamento di un job nello spike server
\end{itemize}

\subsection{Logica di controllo e routing}
Ogni volta che arriva un nuovo job:
\begin{itemize}
    \item Se \(SI \leq SI\_max\), viene incrementato SI di 1 e il job viene inviato al web server.
    \item Se \(SI > SI\_max\), viene incrementato n\_spike di 1 e il job viene inviato allo spike server.
\end{itemize} 

Ogni volta che job termina decremenenta il corrispettivo contatore restituendo il token.

\subsection{Implementazione dello scheduler}
Al contrario di quanto avviene nel caso di studio di riferimento in cui si usa uno scheduling processor sharing, in questo progetto userò uno scheduling FIFO per entrambi i server.


\subsection{Configurazione dei parametri delle distribuzioni}
\begin{itemize}
    \item Stream 0: Inter-arrivi (Iperesponenziale, media 0.15s, cv=4).
    \item Stream 1: Servizio Web Server (Iperesponenziale, media 0.16s, cv=4).
    \item Stream 2: Servizio Spike Server (Iperesponenziale, media 0.08s, cv=4).
\end{itemize}

    




