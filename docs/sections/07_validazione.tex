\section{Validazione}
Dato che in questo studio non sono disponibili dati reali con cui confrontare i risultati ottenuti dalla simulazione, la validazione del modello simulativo confronterà i risultati ottenuti con quelli ottenuti dallo studio di riferimento nella prossima sezione del report.


\subsection{Confronto tempi di risposta}
Nello studio di riferimento vengono effettuati dei test sui tempi di risposta sul webserver e sullo spike server. 

\begin{figure}[H]
    \centering
    \includegraphics[width=1\textwidth]{images/response_time_serazzi.png}
    \caption{Tempi di risposta del webserver e dello spike server in funzione del carico nel caso di studio di Serazzi}
    \label{fig:response_time_serazzi}
\end{figure}

\begin{figure}[H]
    \centering
    \includegraphics[width=1\textwidth]{images/response_time_web_vs_si_max.png}
    \caption{Tempi di risposta del webserver in funzione del SI\_max ottenuti dalla simulazione.}
    \label{fig:response_time_web_sim}
\end{figure}

\begin{figure}[H]
    \centering
    \includegraphics[width=1\textwidth]{images/response_time_spike_vs_si_max.png}
    \caption{Tempi di risposta dello spike server in funzione del SI\_max ottenuti dalla simulazione.}
    \label{fig:response_time_spike_sim}
\end{figure}

\begin{table}[H]
  \begin{center}
    \caption{Validazione tempi di risposta del webserver}
    \label{tab:validazione_response_time_web_vs_si}
    \pgfplotstabletypeset[
      multicolumn names, % allows to have multicolumn names
      col sep=comma, % the seperator in our .csv file
      display columns/0/.style={
        column name={$SI_{max}$},
        column type={S[round-mode=places, round-precision=0]},string type},  % use siunitx for formatting
      display columns/1/.style={
        column name={$E[T_s]_{web}$},
        column type={S[round-mode=places, round-precision=4]},string type},
      display columns/2/.style={
        column name={$E[T_s]_{web}$ CI95},
        column type={S[round-mode=places, round-precision=4]},string type},
      every head row/.style={
        column name={},
        before row={\toprule}, % have a rule at top
        after row={\midrule}, % rule under units
            },
        every last row/.style={after row=\bottomrule}, % rule at bottom
    ]{data_tables/response_time_web_vs_si_max.csv} % filename/path to file
  \end{center}
\end{table}

\begin{table}[H]
  \begin{center}
    \caption{Validazione tempi di risposta dello spike server}
    \label{tab:validazione_response_time_spike_vs_si_max}
    \pgfplotstabletypeset[
      multicolumn names, % allows to have multicolumn names
      col sep=comma, % the seperator in our .csv file
      display columns/0/.style={
        column name={$SI_{max}$},
        column type={S[round-mode=places, round-precision=0]},string type},  % use siunitx for formatting
      display columns/1/.style={
        column name={$E[T_s]_{spike}$},
        column type={S[round-mode=places, round-precision=4]},string type},
      display columns/2/.style={
        column name={$E[T_s]_{spike}$ CI95},
        column type={S[round-mode=places, round-precision=4]},string type},
      every head row/.style={
        column name={},
        before row={\toprule}, % have a rule at top
        after row={\midrule}, % rule under units
            },
        every last row/.style={after row=\bottomrule}, % rule at bottom
    ]{data_tables/response_time_spike_vs_si_max.csv} % filename/path to file
  \end{center}
\end{table}

Controntandoli con i risultati ottenuti dalla simulazione progettata, si nota come l'andamento sia essenzialmente lo stesso, tuttavia i valori dei tempi differiscono. Questo è dovuto dal tempo di simulazione che il caso di studio originale non esplicita direttamente e quindi probabilmente differente dai 5000 secondi utilizzati nelle simulazioni. Tuttavia per non rendere troppo lunghi gli esperimenti, è stata eseguita una simulazione con \(SI{max}\) = 160 e aumentato il tempo di simulazione a 20000 secondi, per dimostrare che il motivo della differenza nei tempi di risposta è dovuto proprio al tempo di simulazione. 

\begin{figure}[H]
    \centering
    \includegraphics[width=1\textwidth]{images/run_20000sec.png}
    \caption{Tempi di risposta del webserver con \(SI_{max} = 160\) ottenuti dalla simulazione con 20000 secondi di simulazione.}
    \label{fig:response_time_web_sim_20000}
\end{figure}

\subsection{Confronto Throughput}
Nello studio di riferimento vengono effettuati dei test sul throughput del webserver.

\begin{figure}[H]
    \centering
    \includegraphics[width=1\textwidth]{images/throughtput_web_spike_serazzi.png}
    \caption{Throughput del webserver e dello spike server in funzione del carico nel caso di studio di Serazzi}
    \label{fig:throughput_serazzi}
\end{figure}

\begin{figure}[H]
    \centering
    \includegraphics[width=1\textwidth]{images/throughput_web_vs_si_max.png}
    \caption{Throughput del webserver in funzione del SI\_max ottenuti dalla simulazione.}
    \label{fig:throughput_web_sim}
\end{figure}

\begin{figure}[H]
    \centering
    \includegraphics[width=1\textwidth]{images/throughput_spike_vs_si_max.png}
    \caption{Throughput dello spike server in funzione del SI\_max ottenuti dalla simulazione.}
    \label{fig:throughput_spike_sim}
\end{figure}

L'andamento e i valori qui sono essenzialmente gli stessi. Purtroppo il caso di studio originale non fornisce i valori numerici precisi su cui ha costruito i grafici per confrontarli direttamente.


