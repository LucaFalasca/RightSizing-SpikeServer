\section{Risultati}
Gli esperimenti condotti mirano a raggiungere gli obiettivi delineati nella sezione 2.
\subsection{Obiettivo 1}
Come si può evincere dalla figura \ref{fig:response_time_vs_si_max}, il tempo di risposta medio \(E[R]\) rimane al di sotto della soglia di 8 secondi per valori di \(SI_{max}\) fino a circa 90. Esattamente come suggerito nello studio di riferimento, il valore ottimale di \(SI_{max}\) che massimizza l'utilizzo del web server mantenendo il rispetto dello SLA si attesta intorno a 80-90. Considerando che con l'intervallo di confidenza con 90 si sfora, il valore più prudente da adottare per \(SI_{max}\) tra i due risulta essere 80.
\begin{figure}[H]
    \centering
    \includegraphics[width=1\textwidth]{images/response_time_vs_si_max.png}
    \caption{Tempo di risposta medio in funzione di \(SI_{max}\)}
    \label{fig:response_time_vs_si_max}
\end{figure}

\begin{table}[H]
    \begin{center}
        \caption{Tempo di risposta medio in funzione di \(SI_{max}\)}
        \pgfplotstabletypeset[
        multicolumn names, % allows to have multicolumn names
        col sep=comma, % the seperator in our .csv file
        display columns/0/.style={
            column name={$SI_{max}$},
            column type={S[round-mode=places, round-precision=0]},string type
        },
        display columns/1/.style={
            column name={$E[T_s]$},
            column type={S[round-mode=places, round-precision=4]},string type
        },
        display columns/2/.style={
            column name={$E[T_s]$ CI95},
            column type={S[round-mode=places, round-precision=4]},string type
        },
        every head row/.style={
            before row=\toprule, % adds a top rule before the table header
            after row=\midrule,  % adds a mid rule after the table header
        },
        every last row/.style={
            after row=\bottomrule, % adds a bottom rule after the table
        },
        ]{../docs/data_tables/response_time_vs_si_max.csv}
    \end{center}
\end{table}
 

\subsection{Obiettivo 2}
Come mostrato in figura \ref{fig:response_time_vs_arrival_rate_vs_si_max}, accade lo stesso fenomeno descritto nello studio di riferimento: al crescere del tasso di arrivo, il tempo di risposta medio aumenta all'aumentare degli arrivi, fino a che non si arriva a superare la soglia \(SI_{max}\). A questo punto, il tempo di risposta comincia a calare fino a che anche lo spike server si satura e il tempo di risposta ricomincia a salire per valori superiori a 11. Nelle tabelle sottostanti sono riportati i valori numerici però mostrando solo i valori di \(SI_{max}\) a multipli di 20 per la grande quantità di dati generata.

\begin{figure}[H]
    \centering
    \includegraphics[width=1\textwidth]{images/response_time_vs_arrival_rate_vs_si_max.png}
    \caption{Tempo di risposta medio in funzione del tasso di arrivo e di \(SI_{max}\)}
    \label{fig:response_time_vs_arrival_rate_vs_si_max}
\end{figure}

\begin{table}[H]
    \begin{center}
        \caption{Tempo di risposta medio in funzione del tasso di arrivo e di \(SI_{max}\)}
        \begin{minipage}[t]{0.48\textwidth}
            \pgfplotstabletypeset[
            multicolumn names, % allows to have multicolumn names
            col sep=comma, % the seperator in our .csv file
            display columns/0/.style={
                column name={SI max},
                column type={S[round-mode=places, round-precision=0]},string type
            },
            display columns/1/.style={
                column name={\(\lambda\) },
                column type={S[round-mode=places, round-precision=0]},string type
            },
            display columns/2/.style={
                column name={$E[T_s]$},
                column type={S[round-mode=places, round-precision=4]},string type
            },
            display columns/3/.style={
                column name={$E[T_s]$ CI95},
                column type={S[round-mode=places, round-precision=4]},string type
            },
            every head row/.style={
                before row=\toprule, % adds a top rule before the table header
                after row=\midrule,  % adds a mid rule after the table header
            },
            every last row/.style={
                after row=\bottomrule, % adds a bottom rule after the table
            },
            ]{../docs/data_tables/response_time_vs_arrival_rate_vs_si_max_1.csv}
        \end{minipage}
        \begin{minipage}[t]{0.48\textwidth}
            \pgfplotstabletypeset[
            multicolumn names, % allows to have multicolumn names
            col sep=comma, % the seperator in our .csv file
            display columns/0/.style={
                column name={SI max},
                column type={S[round-mode=places, round-precision=0]},string type
            },
            display columns/1/.style={
                column name={\(\lambda\) },
                column type={S[round-mode=places, round-precision=0]},string type
            },
            display columns/2/.style={
                column name={$E[T_s]$},
                column type={S[round-mode=places, round-precision=4]},string type
            },
            display columns/3/.style={
                column name={$E[T_s]$ CI95},
                column type={S[round-mode=places, round-precision=4]},string type
            },
            every head row/.style={
                before row=\toprule, % adds a top rule before the table header
                after row=\midrule,  % adds a mid rule after the table header
            },
            every last row/.style={
                after row=\bottomrule, % adds a bottom rule after the table
            },
            ]{../docs/data_tables/response_time_vs_arrival_rate_vs_si_max_2.csv}
        \end{minipage}
        
        
    \end{center}
\end{table}

\subsection{Obiettivo 3}
Nell'obiettivo 3 si va ad duplicare la potenza dello spike server, per verificare se questo evita il saturamento dello spike server e quindi il successivo aumento del tempo di risposta medio. Come mostrato in figura \ref{fig:response_time_vs_arrival_rate_vs_si_max_enhanced_spike}, lo spike server non arriva più a saturarsi e il tempo di risposta continua a diminuire all'aumentare del tasso di arrivo. 


\begin{figure}[H]
    \centering
    \includegraphics[width=1\textwidth]{images/response_time_vs_arrival_rate_vs_si_max_enhanced_spike.png}
    \caption{Tempo di risposta medio in funzione del tasso di arrivo e di \(SI_{max}\) con spike server potenziato}
    \label{fig:response_time_vs_arrival_rate_vs_si_max_enhanced_spike}
\end{figure}

\begin{table}
    \begin{center}
        \caption{Tempo di risposta medio in funzione del tasso di arrivo e di \(SI_{max}\) con spike server potenziato}
        \begin{minipage}[t]{0.48\textwidth}
            \pgfplotstabletypeset[
            multicolumn names, % allows to have multicolumn names
            col sep=comma, % the seperator in our .csv file
            display columns/0/.style={
                column name={SI max},
                column type={S[round-mode=places, round-precision=0]},string type
            },
            display columns/1/.style={
                column name={\(\lambda\) },
                column type={S[round-mode=places, round-precision=0]},string type
            },
            display columns/2/.style={
                column name={$E[T_s]$},
                column type={S[round-mode=places, round-precision=4]},string type
            },
            display columns/3/.style={
                column name={$E[T_s]$ CI95},
                column type={S[round-mode=places, round-precision=4]},string type
            },
            every head row/.style={
                before row=\toprule, % adds a top rule before the table header
                after row=\midrule,  % adds a mid rule after the table header
            },
            every last row/.style={
                after row=\bottomrule, % adds a bottom rule after the table
            },
            ]{../docs/data_tables/response_time_vs_arrival_rate_vs_si_max_enhanced_spike_1.csv}
        \end{minipage}
        \begin{minipage}[t]{0.48\textwidth}
            \pgfplotstabletypeset[
            multicolumn names, % allows to have multicolumn names
            col sep=comma, % the seperator in our .csv file
            display columns/0/.style={
                column name={SI max},
                column type={S[round-mode=places, round-precision=0]},string type
            },
            display columns/1/.style={
                column name={\(\lambda\) },
                column type={S[round-mode=places, round-precision=0]},string type
            },
            display columns/2/.style={
                column name={$E[T_s]$},
                column type={S[round-mode=places, round-precision=4]},string type
            },
            display columns/3/.style={
                column name={$E[T_s]$ CI95},
                column type={S[round-mode=places, round-precision=4]},string type
            },
            every head row/.style={
                before row=\toprule, % adds a top rule before the table header
                after row=\midrule,  % adds a mid rule after the table header
            },
            every last row/.style={
                after row=\bottomrule, % adds a bottom rule after the table
            },
            ]{../docs/data_tables/response_time_vs_arrival_rate_vs_si_max_enhanced_spike_2.csv}
        \end{minipage}
    \end{center}
\end{table}