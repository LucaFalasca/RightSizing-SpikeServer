\section{Caso di Studio}
Il sistema oggetto di studio è un'architettura di data center per un Internet Service Provider, progettata per gestire dinamicamente le fluttuazioni di carico e garantire la Quality of Service (QoS) ottimizzando al contempo l'uso delle risorse.

Il problema principale affrontato è il "right-sizing", ovvero come evitare sia il sovradimensionamento (spreco di risorse) sia il sottodimensionamento (violazione degli SLA e degrado delle prestazioni), specialmente in presenza di fluttuazioni di carico a breve e lungo termine.

L'architettura proposta, come descritto nel caso di studio 6.2 del libro di testo “Performance Engineer”, si basa su un livello di scaling verticale che gestisce i picchi di carico improvvisi e di breve durata. Questo livello introduce uno Spike Server dedicato. Un Load Controller monitora un indicatore di picco (Spike Indicator, SI), definito come il numero di richieste concorrenti in esecuzione su un Web Server.

Il comportamento del sistema seguirebbe quanto descritto:
\begin{itemize}
    \item Quando l'indicatore SI supera una soglia di allarme SI\_max, le nuove richieste in arrivo non vengono più inviate al Web Server congestionato, ma vengono reindirizzate allo Spike Server.
    \item Quando il carico sul Web Server diminuisce e SI scende al di sotto della soglia, il routing delle richieste torna alla normalità.
\end{itemize}