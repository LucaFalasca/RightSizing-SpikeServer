\section{Modello concettuale}
Il modello descritto può essere schematizzato come in figura \ref{fig:modello_schematizzato}.
\begin{figure}[H]
    \centering
    \includegraphics[width=\textwidth]{images/modello_schematizzato.png}
    \caption{Modello concettuale del sistema di autoscaling gerarchico con Spike Server}
    \label{fig:modello_schematizzato}
\end{figure}
I job arrivano e a seconda del livello di pienezza dei webservers.
Nonostante il sistema possa sembrare troppo semplice per una analisi simulativa, ci sono una serie di aspetti che lo rendono complesso e difficilmente modellabile solo matematicamente senza semplificazioni:
\begin{itemize}
    \item Il routing non è probabilistico: il routing dei job non è semplicemente probabilistico ( 40\% su uno e 60\% su un altro), ma dipende strettamente dallo stato del webserver nel momento del routing. Questa complicazione rende molto difficile un'analisi statica, soprattutto nel transiente.
\end{itemize}

Spiegazione tempi di servizio esponenziale: Nel caso di studio affrontato nel libro viene utilizzata una distribuzione iperesponenziale nel tassi di servizio per modellare il fatto che ad un server arrivano job di dimensione molto variabile. Ovviamente questa cosa potrebbe essere modellata anche utilizzando delle size dei job differenti anziché agire sul tasso di servizio. Tuttavia ho deciso di attenermi al testo originale e utilizzare anche io dei tassi di servizio iperesponenziali per modellare questo comportamento. 

\subsection{Spike server}
Lo Spike Server avrà una potenza maggiore rispetto al webserver. In particolare avrà un tasso di servizio doppio rispetto a quello del webserver principale, in modo da poter smaltire rapidamente i picchi di carico. Per il calcolo di SI\_max è importante sottolineare che lo spike server è bene usarlo il meno possibile, in quanto è una risorsa più costosa rispetto al webserver. Quindi anziché cercare di minimizzare il tempo di risposta medio del sistema, si cerca di rispettare lo SLA (8 secondi) utilizzando il meno possibile lo spike server.

\subsection{Web Server}
Il web server sarà solamente uno, per semplicità di modellazione. Per questo motivo non vengono gestite le fluttuazioni a lungo termine, in quanto queste richiederebbero un sistema con più webserver per utilizzare una scaling orizzontale. 